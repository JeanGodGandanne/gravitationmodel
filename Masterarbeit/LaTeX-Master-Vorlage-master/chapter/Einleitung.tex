\chapter{Einleitung}
\section{Einführung}
Im Jahr 2019 gaben private Haushalte in Deutschland rund 197,3 Milliarden Euro für Lebensmittel aus. \cite{statista2019Ausgaben}
Dies verteilt sich auf rund 34.947 Geschäfte zwar kontrollieren die vier Unternehmen Edeka, Rewe, Aldi und die Schwarz-Gruppe rund 70 Prozent des Marktes, dennoch herrscht ein Konkurrenzkampf der jeden noch so kleinen Vorteil gegenüber den Wettbewerbern in direkten Umsatz, Marktanteil oder Kundenzufriedenheit Anstieg nieder spiegelt.  \cite{statista2018Geschäfte}
Das Spielfeld ist hierbei vielfältig und wechselhaft, Faktoren ändern sich. In Zeiten der fortgeschrittenen Digitalisierung sind Unternehmen mittlerweile auf technische Unterstützung angewiesen, um nicht abgehängt zu werden. Sämtliche Bereiche der Maschinerie Lebensmitteleinzelhandel (im Folgenden LEH) sind teilweise vollständig oder zu großen Teilen von technischen Prozessen durchzogen.
Dennoch gibt es LEH bereits seit Jahrzehnten und gängige Prozesse haben sich etabliert und verbreitet. So auch die Standortanalyse und die Filialplanung. Eine spannende Aufgabe der Unternehmen und der Wissenschaft ist es nun neue Technologien für die digitale Umsetzung der gegebenen Prozesse zu nutzen und durch das Zusammenspiel Optimierung und Innovation zu erreichen. 

\section{Motivation}
Viele der in der Standortanalyse verwendeten Prozesse und Algorithmen sind geo-mathematischer oder geo-informatischer Natur und liegen teils komplexer Berechnungen zu Grunde.
Diese zu erkunden und ergründen haben sich eigene Wissenschaftsbereiche aus der Mathematik, Betriebswirtschaftslehre, Informatik und dem Ingenieurswesen gebildet, die diverse Studiengänge und Ausbildungen beheimaten.
Um diese komplexen Themen zu vereinfachen und den Mitarbeitern des LEH die tägliche Arbeit zu erleichtern, können digitale Prozesse eingesetzt werden.
In benutzerfreundlichen, einfach zu verstehenden und visuell ansprechenden Anwendungen sollen sich die Algorithmen und Prozesse der Standortanalyse und Filialplanung verbergen. 
Solche Anwendungen werden Geo-Informationssysteme (kurz und im Folgenden GIS) genannt.
Auf dem Markt existieren derer bereits einige.
Big Player sind zum Beispiel Pitney Bowes, ESRI, oder Autodesk.
Ebenso existieren einige OpenSource Angebote wie zum Beispiel GRASS GIS oder QGIS.
In der Praxis benötigen die Unternehmen dennoch oft Individuallösungen, die entweder auf bestehender Software aufbauen oder diese integrieren.
Ziel dieser Arbeit ist es daher eine Anwendung zu entwickeln, die Algorithmen und Prozesse der Standortanalyse und Filialplanung einfach implementiert und in eine solche Individuallösung integriert werden kann.
Zu dieser Anwendung gehören also lediglich eine Karte, Karten-Werkzeuge und Geo-Objekte als Teilobjekte eines kompletten GIS.


\section{Abgrenzung}
In dieser Arbeit werden Themenfelder des Geo-Marketings, der Geo-Informatik und -Mathematik sowie diverse Technologien behandelt. 
Die resultierende Anwendung soll nichts weiter als ein Prototyp darstellen und ist keinesfalls ein komplettes GIS. 
Vielmehr wird sich auf die Ausarbeitung des Huff-Models zu Standortanalyse und Filialplanung konzentriert mit dem Fokus auf Umsetzbarkeit innerhalb einer Web-Anwendung mit OpenSource Technologien. 