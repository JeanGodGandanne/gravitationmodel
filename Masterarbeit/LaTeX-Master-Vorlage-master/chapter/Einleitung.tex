\chapter{Einleitung}
Bevor die theoretischen Grundlagen der Arbeit vorgestellt werden, geben die folgenden Kapitel einen Einstieg in die Thematik und beschreiben, was den Lesenden in diesem Dokument erwarten.

\section{Einführung}
Im Jahr 2019 gaben private Haushalte in Deutschland rund 197,3 Milliarden Euro für Lebensmittel aus \footcite{statista_konsumausgaben}.
Dies verteilt sich auf rund 34.947 Geschäfte.
Zwar kontrollieren die vier Unternehmen Edeka, Rewe, Aldi und die Schwarz-Gruppe rund 70 Prozent des Marktes, dennoch herrscht ein Konkurrenzkampf der jeden noch so kleinen Vorteil gegenüber den Wettbewerbern direkt in einen Anstieg in Umsatz, Marktanteil oder Kundenzufriedenheit nieder spiegelt \footcite{statista_lebensmittel-discounter}.\\
Das Spielfeld ist hierbei vielfältig und wechselhaft, Faktoren ändern sich. 
In Zeiten der fortgeschrittenen Digitalisierung sind Unternehmen mittlerweile auf technische Unterstützung angewiesen, um nicht abgehängt zu werden. 
Sämtliche Bereiche der Maschinerie Lebensmitteleinzelhandel (im Folgenden LEH) sind teilweise vollständig oder zu großen Teilen von technischen Prozessen durchzogen.
Dennoch gibt es den LEH bereits seit Jahrzehnten und gängige Prozesse haben sich etabliert und verbreitet. 
So auch die Standortanalyse und die Filialplanung. 
Eine spannende Aufgabe der Unternehmen und der Wissenschaft ist es, stetig neue Technologien für die digitale Umsetzung der gegebenen Prozesse zu nutzen und durch das Zusammenspiel Optimierung und Innovation zu erreichen. \\
Kennzahlen im Marketing die in Zusammenhang mit dem Umsatz stehen sind unter Anderem Marktanteil und die Anzahl der Kunden.
Diese gilt es zu erhöhen, möchte man mehr Umsatz machen.
Zwar ist diese Gleichung vereinfacht und unvollständig, dennoch gibt sie einen guten Einblick was in der Standortanalyse und Filialplanung primär verglichen und  versucht zu erreichen wird.\\
Wie kann die Anzahl der Kunden und der Marktanteil erhöht werden?\\
Eine Lösung liegt in der Eröffnung neuer Filialen - Expansion zur Erschließung neuer Märkte oder Verstärkung der Präsenz in bestehenden Märkten sind hierbei Möglichkeiten.
Bei der Planung einer neuen Filiale kommt es auf viele Faktoren an.
Potenzielle Standorte müssen anhand ihres Einflusses auf den Marktanteil der eigenen Filialen sowie der Konkurrenz verglichen und bewertet werden.
Nicht immer ist es am sinnvollsten den Standort der neuen Filiale so zu wählen, dass die Konkurrenz Marktanteile verliert.
Denn möglicherweise treten Kannibalisierungseffekte auf, bei denen es auch bei anderen Filialen der eigenen Kette zu Umsatzverlusten kommt.
Es ist stets ein Gesamtüberblick zu wahren und den Einfluss der neuen Filiale auf den Gesamtmarktanteil zu analysieren.\\
Um eine Abwägung und Rangliste der Standorte erstellen zu können, muss ein Indikator errechnet werden.
In die Berechnung fallen viele Variablen und Parameter, die sich grundsätzlich in zwei Kategorien unterteilen:\\
Faktoren bezogen auf die Filiale und Faktoren bezogen auf den Kunden.
Faktoren dieser Kategorien haben Einfluss auf die Entscheidung eines Kunden bei einer Filiale an einem gewissen Standort einkaufen zu gehen.
Denn im Grunde geht es genau darum.
Warum und mit welcher Wahrscheinlichkeit entscheidet sich ein Kunde bei einer Filiale einkaufen zu gehen?\\
In der Wissenschaft wird hierbei von Gravitationsmodellen gesprochen.
Jede Filiale besitzt eine Art Anziehungskraft auf Kunden.
Eigenschaften wie zum Beispiel Größe, Preiskultur oder Markenstärke der Filiale sowie die Entfernung und Reisezeiten des Kunden zur Filiale bestimmen die Gravitation einer Filiale.
Ebenso beeinflussen sich Filialen, der eigenen Kette wie auch Konkurrenzfilialen, gegenseitig.
Dies erschafft ein Modell eines Filialnetzes, in dem verschiedene Gravitationsebenen der gleichen Stärke zu Isolinien zusammengefasst werden können.
Farblich gefärbt fügt sich das alles zu einem Bild zusammen, welches einen Überblick über die Aufteilung des Marktes gibt.
Es ist gut ersichtlich, welche Filialen häufig besucht werden und welche weniger Besucher anlocken.
Wird in das bestehende Bild eine neue Filiale gesetzt, verschieben sich die Gravitationsebenen entsprechend der Neuberechnung der Besuchswahrscheinlichkeiten.
Erfolgt dies für sämtliche potenzielle neue Standorte,  können die Besuchswahrscheinlichkeiten verglichen und der Marktanteileinfluss analysiert werden.
Dann kann eine Rangliste der Standorte erstellt und der optimale Standort für eine neue Filiale gewählt werden.

\section{Motivation}
Viele der in der Standortanalyse verwendeten Prozesse und Algorithmen sind geomathematischer oder geoinformatischer Natur und liegen teils komplexer Berechnungen zu Grunde.
Diese zu erkunden und ergründen haben sich eigene Wissenschaftsbereiche aus der Mathematik, Betriebswirtschaftslehre, Informatik und dem Ingenieurswesen gebildet, die diverse Studiengänge und Ausbildungen beheimaten.
Um diese komplexen Themen zu vereinfachen und den Mitarbeitern des LEH die tägliche Arbeit zu erleichtern, können digitale Prozesse eingesetzt werden.
In benutzerfreundlichen, einfach zu verstehenden und visuell ansprechenden Anwendungen sollen sich die Algorithmen und Prozesse der Standortanalyse und Filialplanung verbergen. 
Solche Anwendungen werden Geo-Informationssysteme (kurz und im Folgenden GIS) genannt.
Auf dem Markt existieren derer bereits einige.
Big Player sind zum Beispiel Pitney Bowes, ESRI, oder Autodesk.
Ebenso existieren einige OpenSource Angebote wie zum Beispiel GRASS GIS oder QGIS.
In der Praxis benötigen die Unternehmen dennoch oft Individuallösungen, die entweder auf bestehender Software aufbauen oder diese integrieren.
Ziel dieser Arbeit ist es daher eine Anwendung zu entwickeln, die den Algorithmus des Huff-Modells mit Prozessen der Standortanalyse und Filialplanung verbindet und in eine solche Individuallösung einfach integriert werden kann.


\section{Abgrenzung}
In dieser Arbeit werden Themenfelder des Geomarketings, der Geoinformatik und Geomathematik sowie diverse Technologien behandelt. 
Die resultierende Anwendung soll nichts weiter als ein Prototyp darstellen und ist keinesfalls ein komplettes GIS. 
Das Themengebiet Standortanalyse und Filialplanung wird in allgemeiner Form vorgestellt, jedoch wird sich vielmehr auf die Ausarbeitung des Huff-Modells als Algorithmus zum mathematischen Vergleich verschiedener Standorte potenzieller neuer Filialen eines Einzelhändlers fokussiert.
Die Implementierung hat den Fokus auf Umsetzbarkeit innerhalb einer Web-Anwendung mit OpenSource Technologien gelegt. 