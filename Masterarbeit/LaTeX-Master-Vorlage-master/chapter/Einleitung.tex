\chapter{Einleitung}
Bevor die theoretischen Grundlagen der Arbeit vorgestellt werden, geben die folgenden Kapitel einen Einstieg in die Thematik und beschreiben, was den Lesenden in diesem Dokument erwartet.

\section{Einführung}
Im Jahr 2019 gaben private Haushalte in Deutschland rund 197,3 Milliarden Euro für Lebensmittel aus \footcite{statista_konsumausgaben}.
Dies verteilt sich auf rund 34.947 Geschäfte.
Zwar kontrollieren die vier Unternehmen Edeka, Rewe, Aldi und die Schwarz-Gruppe rund 70 Prozent des Marktes, dennoch herrscht ein Konkurrenzkampf, der jeden noch so kleinen Vorteil gegenüber den Wettbewerbern direkt in einem Anstieg in Umsatz, Marktanteil oder Zufriedenheit der Kundschaft ausdrückt \footcite{statista_lebensmittel-discounter}.\\
Das Spielfeld ist hierbei vielfältig und wechselhaft, Faktoren ändern sich. 
In Zeiten der fortgeschrittenen Digitalisierung sind Unternehmen mittlerweile auf technische Unterstützung angewiesen, um nicht abgehängt zu werden. 
Sämtliche Bereiche der Maschinerie Lebensmitteleinzelhandel (im Folgenden LEH) sind teilweise vollständig oder zu großen Teilen von technischen Prozessen durchzogen.
Dennoch gibt es den LEH bereits seit Jahrzehnten und gängige Prozesse haben sich etabliert und verbreitet. 
So auch die Standortanalyse und die Filialplanung. 
Eine spannende Aufgabe der Unternehmen und der Wissenschaft ist es, stetig neue Technologien für die digitale Umsetzung der gegebenen Prozesse zu nutzen und durch das Zusammenspiel Optimierung und Innovation zu erreichen. \\
Kennzahlen im Marketing, die in Zusammenhang mit dem Umsatz stehen, sind unter anderem der Marktanteil und die Anzahl der Kund\_innen.
Diese gilt es zu erhöhen, wenn eine Umsatzsteigerung erwünscht ist. 
Zwar ist diese Gleichung vereinfacht und unvollständig, dennoch gibt sie einen guten Einblick davon, was in der Standortanalyse und Filialplanung primär verglichen und zu erreichen versucht wird.\\
Es stellt sich die Frage, wie die Anzahl der Kund\_innen und der Marktanteil erhöht werden kann.\\
Eine mögliche Lösung liegt in der Eröffnung neuer Filialen, beispielsweise durch die Expansion zur Erschließung neuer Märkte oder die Verstärkung der Präsenz in bestehenden Märkten.
Bei der Planung einer neuen Filiale sollten viele Aspekte näher beleuchtet werden.
Potenzielle Standorte müssen anhand ihres Einflusses auf den Marktanteil der eigenen Filialen sowie der Konkurrenz verglichen und bewertet werden.
Nicht immer ist es am sinnvollsten den Standort der neuen Filiale so zu wählen, dass die Konkurrenz Marktanteile verliert.
Denn möglicherweise treten Kannibalisierungseffekte auf, bei denen es auch bei anderen Filialen der eigenen Kette zu Umsatzverlusten kommt.
Es ist stets ein Gesamtüberblick zu wahren und den Einfluss der neuen Filiale auf den Gesamtmarktanteil zu analysieren.\\
Um eine Abwägung und Rangliste der Standorte erstellen zu können, muss ein Indikator errechnet werden.
In die Berechnung fallen viele Variablen und Parameter, die sich grundsätzlich in zwei Kategorien unterteilen:\\
Faktoren bezogen auf die Filiale und Faktoren bezogen auf die Kund\_innen.
Faktoren dieser Kategorien haben Einfluss auf die Entscheidung der Kund\_innen bei einer Filiale, an einem gewissen Standort, einkaufen zu gehen.
Die Wissenschaft untersucht hierbei mögliche Gründe, wonach die Filiale ausgesucht wird und wie hoch die Wahrscheinlichkeit eines Besuchs ist. 
Hierbei wird von sogenannten Gravitationsmodellen Gebrauch gemacht.\\
Jede Filiale besitzt eine Anziehungskraft auf die Kund\_innen.
Eigenschaften wie zum Beispiel Größe, Preiskultur oder Markenstärke der Filiale sowie die Entfernung und Reisezeiten der Kund\_innen zur Filiale bestimmen die Gravitation dieser.
Ebenso beeinflussen sich Filialen der eigenen Kette, wie auch Konkurrenzfilialen gegenseitig.
Dies erschafft ein Modell eines Filialnetzes, in dem verschiedene Gravitationsebenen der gleichen Stärke zu  Linien zusammengefasst werden können, bei diesen sogenannten Isolinien werden benachbarte Punkte gleichen Wertes verbunden.  \footcite{neumair_definition_2018}.
Farblich gekennzeichnet fügen sich diese Ebenen zu einem Bild zusammen, welches einen Überblick über die Aufteilung des Marktes gibt \footnote{Weitere Informationen zum Gravitationsmodell und dessen Umsetzung werden im späteren Verlauf der Arbeit näher erläutert.}.
Es ist gut ersichtlich, welche Filialen häufig besucht werden und welche weniger Besuchende anlocken.
Wird in das bestehende Bild eine neue Filiale gesetzt, verschieben sich die Gravitationsebenen entsprechend der Neuberechnung der Besuchswahrscheinlichkeiten.
Erfolgt dies für sämtliche potenzielle neue Standorte,  können die Besuchswahrscheinlichkeiten verglichen und der Marktanteileinfluss analysiert werden.
Abschließend kann eine Rangliste der Standorte erstellt und der optimale Standort für eine neue Filiale gewählt werden.

\section{Motivation}
Viele der in der Standortanalyse verwendeten Prozesse und Algorithmen sind geomathematischer oder geoinformatischer Natur und basieren auf komplexen Berechnungen.
Um diese zu erkunden und zu ergründen, haben sich eigene Wissenschaftsbereiche in der Mathematik, Betriebswirtschaftslehre, Informatik und dem Ingenieurswesen gebildet.
Zur Vereinfachung dieser komplexen Themen und der daraus resultierenden Erleichterung der täglichen Arbeit für die Mitarbeitenden des LEH, können digitale Prozesse eingesetzt werden.
In benutzerfreundlichen, einfach zu verstehenden und visuell ansprechenden Anwendungen sollen sich die Algorithmen und Prozesse der Standortanalyse und Filialplanung verbergen. 
Solche Anwendungen werden Geoinformationssysteme (kurz und im Folgenden GIS) genannt.
Auf dem Markt existieren derer bereits einige.
Marktführer sind zum Beispiel Pitney Bowes, ESRI, oder Autodesk.
Ebenso existieren einige OpenSource Angebote, wie zum Beispiel GRASS GIS oder QGIS.
In der Praxis benötigen die Unternehmen dennoch oft Individuallösungen, die entweder auf bestehender Software aufbauen oder diese integrieren.
Ziel dieser Arbeit ist es daher eine Anwendung zu entwickeln, die den Algorithmus des Huff-Modells mit Prozessen der Standortanalyse und Filialplanung verbindet.


\section{Abgrenzung}
In dieser Arbeit werden Themenfelder des Geomarketings, der Geoinformatik und Geomathematik sowie diverse Technologien behandelt. 
Die resultierende Anwendung soll lediglich einen Prototyp darstellen und ist keinesfalls ein komplettes GIS. 
Der Prototyp umfasst eine Karte, Layer zu Filialen und Zensusgebieten, eine einfache Layerverwaltung sowie Funktionalitäten zur Berechnung des Huff-Modells und grafischen Darstellung der Gravitationsebenen.
Die Berechnung erfolgt anhand von realitätsnahen Beispieldaten, eine reale Anwendung ist nicht vorgesehen. 
Die Themengebiete Standortanalyse und Filialplanung werden in allgemeiner Form vorgestellt, jedoch wird sich vielmehr auf die Ausarbeitung des Huff-Modells als Algorithmus zum mathematischen Vergleich verschiedener Standorte potenzieller neuer Filialen eines Einzelhändlers fokussiert.
Die Implementierung hat den Fokus auf Umsetzbarkeit innerhalb einer Web-Anwendung mit OpenSource Technologien gelegt. 