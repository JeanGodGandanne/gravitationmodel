\chapter*{Abstract}
Die Besuchswahrscheinlichkeit eines Kunden hat großen Einfluss auf den Umsatz und Marktanteil einer Filiale.
Große Verkaufsketten des Lebensmitteleinzelhandels stehen hierzu in ständigem Konkurrenzkampf.
Neben einer eigenen Preispolitik oder dem individuellen Sortiment, kann vor allem die Eröffnung einer weiteren Filiale Einfluss auf die Besuchswahrscheinlichkeit haben.
Bei der Wahl des Standortes sollten mehrere Faktoren in Betracht gezogen werden.
Die Berechnung der Wahrscheinlichkeit kann über Gravitationsmodelle  erfolgen.\\
Eines der bekanntesten Modelle wurde von David L. Huff in 1962 entwickelt.
Es hat den Zahn der Zeit überstanden und ist nach wie vor aufgrund der einfachen Anwendung und dennoch großen Aussagekraft sehr angesehen und wurde stetig weiterentwickelt. 
Für die digitale Verwendung als Komponente eines Web-Geoinformationssystems (Web-GIS) wird dieses Modell mit moderner Technologie, in Form von Angular und OpenLayers, prototypisch implementiert und anschließend die Nutzbarkeit in realen Szenarien bewertet.

The probability of a customer visiting a store has a major influence on its sales and market share.
Large food retail chains are in constant competition with each other in this respect.
In addition to their own pricing policy or individual product range the opening of a further store in particular can have an influence on the probability of a customer visiting the store.
When choosing a location several factors should be taken into account.
The calculation of the probability can be done by gravity models.
One of the best known models was developed by David L. Huff in 1962.
It has survived the ravages of time and is still highly regarded due to its ease of use and yet great predictive power and has been continually refined. 
For digital use as a component of a web geographic information system (Web-GIS) this model is prototyped using modern technology, in the form of Angular and OpenLayers, and then its usability in real-world scenarios is evaluated.