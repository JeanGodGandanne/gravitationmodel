\chapter*{Abstract}
Die Besuchswahrscheinlichkeit eines Kunden hat großen Einfluss auf den Umsatz und Marktanteil einer Filiale.
Großen Verkaufsketten des Lebensmitteleinzelhandels stehen hierzu in ständigem Konkurrenzkampf.
Neben der eigene Preispolitik oder individuellem Sortiment, kann vor Allem über die Eröffnung einer weiteren Filiale Einfluss auf Die Besuchswahrscheinlichkeit genommen werden.
Bei der Wahl des Standortes sollten mehrere Faktoren in Betracht gezogen werden.
Die Berechnung der Wahrscheinlichkeit kann über Gravitationsmodelle  erfolgen.\\
Eines der bekanntesten Modelle wurde von David L. Huff in 1962 entwickelt.
Es hat den Zahn der Zeit überstanden und ist nach wie vor aufgrund der einfachen Anwendung und dennoch großen Aussagekraft sehr angesehen und hat viele Erweiterungen hervorgebracht.
Für die digitale Anwendung als Komponente eines Web-Geoinformationssystems (Web-GIS) wird dieses Modell mit moderner Technologie, in Form von Angular und OpenLayers, prototypisch implementiert und deren Anwendung bewertet.

The probability of a customer visiting a store has a major influence on its sales and market share.
Large food retail chains are in constant competition with each other in this respect.
In addition to the company's own pricing policy or individual product range, the likelihood of a customer visiting a store can be influenced above all by the opening of a further store.
When choosing a location, several factors should be taken into account.
The calculation of the probability can be done by gravity models.
One of the best known models was developed by David L. Huff in 1962.
It has survived the ravages of time and is still highly regarded for its ease of use yet great predictive power, and has spawned many extensions.
For digital application as a component of a web geographic information system (web GIS), this model is prototyped using modern technology, in the form of Angular and OpenLayers, and its application is evaluated.
