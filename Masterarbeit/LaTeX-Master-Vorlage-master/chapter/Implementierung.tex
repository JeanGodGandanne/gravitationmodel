\chapter{Implementierung}
Das Kapitel Implementierung erfasst die technische Dokumentation des erarbeiteten Prototyps.
Im Detail werden in den folgenden Seiten die technische Umsetzung des in Kapitel \ref{ch:concept} entworfenen Konzepts beschrieben. 
Beschriebene Funktionalitäten werden mit Bildausschnitten unterstützt.

\section{Prototyp}
Die Anwendung wurde als Angular Projekt mittels der Angular CLI erstellt. 
Über den CLI Befehl 

\begin{lstlisting}[language=bash]
$ ng new gravitationsmodel
\end{lstlisting}

generiert die CLI ein kompilierbares und ausführbares Angular Projekt mit essentiellen Abhängigkeiten und Strukturen.
Angular verwendet standardmäßig NPM als package manager.
Eine \emph{package.json}, welche sämtliche Abhängigkeiten dokumentiert, wird bereits mit erstellt.
Um das Setup abzuschließen müssen weitere Abhängigkeiten in Form von \emph{npm packages} installiert werden.
Über den Befehl 

\begin{lstlisting}[language=bash]
$ ng add ol
\end{lstlisting}

fügt die CLI automatisch das Paket von OpenLayers hinzu und für die korrekte Typisierung in TypeScript das notwendige \emph{types} Paket für OpenLayers hinzu.

Nachdem die technischen Voraussetzungen geschaffen sind kann mit der Implementierung begonnen werden.
Auch hierbei bietet die CLI Unterstützung in Form von \emph{Scaffolding} Befehlen.

\begin{lstlisting}[language=bash]
$ ng generate <schematic> [name]
\end{lstlisting}

generiert Komponenten, Services, Models, Klassen und weitere Code-Gerüste durch den passenden Präfix-Parameter und Namen.
Angular Komponenten bestehen immer aus einer TypeScript Klasse (\emph{.component.ts}), einem Template (\emph{.component.html}) sowie Dateien für Styling (\emph{.css} oder \emph{.scss}) und Test (\emph{.component.spec.ts}).

\begin{lstlisting}[language=bash]
$ ng generate component objektfenster
\end{lstlisting}

erstellt \emph{objektfenster.component.ts}, \emph{objektfenster.component.html} sowie \emph{objektfenster.scss} und \emph{objektfenster.component.spec.ts} Dateien.

Mit Hilfe der CLI lassen sich so die Code-Gerüste schnell erstellen und die CLI übernimmt sogar die in Angular nötige \emph{Dependency Injection} indem neue Komponenten direkt in der entsprechenden \emph{.module.ts} deklariert werden \footcite{angular_cli}.

Nachdem die Komponenten erstellt wurden kann nun mit der Implementierung der fachlichen Funktionen begonnen werden.
Der Kern der Anwendung ist eine Karte mit der interagiert wird.
Um diese darzustellen, muss zunächst ein OpenLayers Map-Objekt erstellt und in den DOM eingehängt werden \footcite{openlayers_map}.
Dies geschieht in der OpenLayersMap-Komponente und dem BaseMap-Service.
Über das bereits installierte OpenLayers Paket wird das Map-Objekt importiert und initialisiert.

\begin{lstlisting}[language=JavaScript]
	import Map from 'ol/Map';
	import {fromLonLat} from 'ol/proj';
	import {View} from 'ol';

	const targetId = 'map';
	const longitude = 13.451338;
	const latitude = 52.503707;
	const zoomLevel = 11;
	
	const coordinate = fromLonLat([longitude, latitude]);
	
	const view = new View({
		center: coordinate,
		zoom: zoomLevel
	});
	
	new Map({
		target: targetId,
		layers: [],
		controls: [],
		view,
	});


\end{lstlisting}

Die Karte wird mit der notwendigen Id des HTML-Elements, in welchem die Karte angezeigt werden soll, und einem View initialisiert.
Das View-Objekt besteht aus einer Center-Koordinate und der initialen Zoomstufe und stellt den Begrenzungsrahmen (engl. Bounding Box) der Karte dar.
Zu beachten ist ebenfalls hierbei die Transformation der Koordinaten von EPSG:4326 in das in OpenLayers standardmäßig genutzte EPSG:3857 über die Funktion \emph{fromLonLat}