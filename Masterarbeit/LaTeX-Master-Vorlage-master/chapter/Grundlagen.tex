\chapter{Grundlagen}
\section{Standortanalyse}
Allgemein erörtert die Standortanalyse die Frage nach einem geeigneten Standort einer Immobilie eines Unternehmens. Es werden verschiedene Faktoren untersucht mit dem Ziel mögliche Standorte auf möglichst wenig Alternativen zu begrenzen. 

Gerade in der Branche des Einzelhandels ist der Standort einer Filiale erfolgsentscheidend. 
Unterschieden wird bei den Faktoren zwischen Makrostruktur und Mikrostruktur.
Die Makrostruktur bezieht sich auf Faktoren die über die einfache Immobilie hinaus gehen. Hierzu zählen das Land, die Region, die Stadt, das Gebiet, in welchem sich der Standort befindet.
Die Mikrostruktur fokussiert Faktoren in direktem Zusammenhang mit der Immobilie. Hierzu zählen das Gebäude, der Zustand, das Aussehen, die Räumlichkeiten und Ausstattung.
Als Oberkategorien der Faktoren einer Standortanalyse für den Einzelhandel können folgende Begriffe abgegrenzt werden:
\begin{itemize}
	\item Bedarf/ Nachfrage
	\item Wettbewerb
	\item Kaufkraft
\end{itemize}

Weiterführend den einzelnen Punkten zuzuordnen sind:
\begin{itemize}
	\item Zielgruppe - Wer sind die Kunden? Ist das Verkaufsmodell B2C oder B2B?
	\item Demografische Merkmale - Alter, Geschlecht und Wohnort der Kunden
	\item Sozioökonomische Betrachtung - Gibt es einen Zusammenhang zwischen Beruf, Bildungsstand oder Einkommen und dem Kaufverhalten?
	\item Psychografische Merkmale - Hat der Lebensstil, Werte, Motivation oder Ähnliches Einfluss auf das Kaufverhalten?
	\item Wettbewerbsdichte
	\item Preispolitik der Wettbewerber
	\item Größe des Einzugsgebiets
	\item Kaufkraftwerte des Gebiets
	\item Erreichbarkeit
	\item Mietpreise
	\item Attraktivität des Standortes
\end{itemize}

Über psychografische Unterscheidungen werden die Kunden in bestimmte Käufer-Kategorien eingestuft.
Hierzu zählen:
\begin{itemize}
	\item Impulsive Käufer - Spontane Entscheidung für den Kauf ohne vorherige Planung und Vergleich
	\item Extensive Käufer - Rational abgewägte Entscheidungen die von bestimmten emotionalen Aspekten beeinflusst werden
	\item Habituelle Käufer - Geplante Entscheidungen ohne Preis- und Qualitätsvergleich vorab
\end{itemize}

Die Kategorien müssen anhand der eigenen Marke als mehr oder weniger relevant bewertet und gewichtet werden, bevor die mathematische Analyse durchgeführt werden kann.
Außerdem müssen nun die einzelnen Kategorien für jeden Standort bewertet werden. 
Da Bewertung der Kategorie muss nun mit der Gewichtung der Kategorie multipliziert werden, um die genaue Faktorbewertung des Standorts für jeden Kategorie zu erhalten. Die Faktorbewertungen summiert ergeben nun Standortbewertung. Die Standortbewertungen gegenübergestellt stellen die Standortanalyse abschließend gelistet dar.

\section{Filialplanung}
Obwohl Filialen geographisch sowie aus betriebswirtschaftlicher Sicht Unternehmen-Standorte sind, wird die Filialplanung von der Standortanalyse unterschieden. So sind Filialen mehrere Standorte, die oft zu einem Filialnetz zusammengefasst werden, welches bei der Analyse berücksichtigt werden muss.
Bei der Filialplannung eingesetzte neue Faktoren sind:
\begin{itemize}
	\item Bekanntheitsgrad, Markenstärke
	\item Freies Potential
	\item Kannibalisierungs-Effekte
	\item Logistikkosten
\end{itemize}
\section{Geoinformationssysteme}
Geoinformationssysteme, kurz GIS, sind Informationssysteme zur Erfassung, Bearbeitung, Organisation, Analyse und Präsentation räumlicher Daten.  
Ähnlich anderer Informationssysteme besteht ein GIS aus Hardware (Computer, Server, Drucker etc.), Software (GIS-Anwendung) und räumlichen Daten (Koordinatensystem, Karten, Geometrien etc.). 
Gegebenenfalls wird die Liste um eine Verwaltungsebene ergänzt sobald die Daten und Funktionen des GIS Rollen und Rechte behaftet sind.
Erste GIS Systeme stammen aus den sechziger Jahren (Canada Geographic Information System \cite{esriCGIS}). 
Zu den ersten Nutzern der Systeme gehörten vor Allem Behörden und Universitäten, so wurden viele grundlegende theoretische Konzepte an der Harvard University von Professor Howard Fisher aufgestellt \cite{fisher1979}.
Mittlerweile haben sich viele Web-GIS etabliert, hierbei wird das Informationssystem über eine Website veröffentlicht oder benutzt. 
Zu den größten Vertretern moderne Web-GIS zählen vor Allem Google Maps, Bing Maps, OpenStreetMaps als OpenSource-Alternative oder etwas kleinere Anbieter wie HERE oder Yandex.Maps.
Kommerzielle Web-Gis bieten meist ein eingeschränktes Funktions-Set, was sie nicht als Produkt für individuelle Software-Lösungen für Firmen in Frage kommen lässt daher greifen viele Firmen auf kommerzielle Desktop-GIS zurück oder lassen sich ganz individuelle Systeme bauen, die aus Web- und Desktop-GIS bestehen.
Zu den bekanntesten GIS zählen Produkte von ESRI, Autodesk, Pitney Bowes oder CAIGOS. 
\section{Genutzte Technologie}
Für die prototypische Ausarbeitung des Modells wurde auf modernste Technologien und Frameworks zugegriffen. So wird als Karten-Framework /gls(ol) in der Version 6.4.3 verwendet sowie /gls(angular) in der Version 10.

OpenLayers bietet ein breites Portfolio an Funktionen und Features passend für sämtliche Anforderungen realer Geo-Informationssysteme die in der Praxis verwendet werden. Im Bezug auf die prototypische Anwendung dieser Arbeit sind vor Allem die Unterstützung der Darstellung von Punkten (Filialen) und der Gravitations-Gebiete in GeoJSON-Format auf verschiedenen Layern notwendig sowie die einfache Konfigurierbarkeit von Karten-Interaktion wie zum Beispiel das Platzieren, Verschieben und Editieren von Punkt-Objekten (Filialen) auf der Karte.

Weiterführend soll die Anwendung schnell, kompakt und modern sein, um die Relevanz aus Performance Gründen auf dem Markt gewährleisten zu können. Daher erfolgt die Umsetzung mit dem auf TypeScript basierenden Framework Angular. Besonderes Augenmerk liegt hierbei im Prototypen eigentlich nur auf der simplen, kompakten und schnellen Auslieferung eines Web-Servers als Host der Anwendung. Zu den für die weiterführende Entwicklung relevant werdenden Features des Frameworks zählen eine große Community, einen fortlaufenden Support sowie eine fortlaufende Entwicklung durch Google, die Verwendung von TypeScript einem Superset von JavaScript mit Verbesserter Funktionalität und sämtlichen Features, die für Entwicklung einer effizienten und anspruchsvollen Single-Page-Webanwendung benötigt werden. 