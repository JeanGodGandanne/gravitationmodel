%Dokumenteinstellungen und Anpassungen
\input{resources/settings/settings}
%Anpassung der Überschriften
\addtokomafont{disposition}{\rmfamily}

%Zusätzliche Farben
\definecolor{darkgreen}{RGB}{0,100,0}

%Umbenennungen
\renewcommand{\lstlistlistingname}{Quelltextverzeichnis}

%Pluszeichen in der Referenc beim zitieren ausblenden
\renewcommand*{\labelalphaothers}{}

%Anpassugen zur Quelltextdarstellung, kann bei Bedarf überschrieben werden (z.B. wenn unterschiedliche Sprachen zum Einsatz kommen)
\renewcommand{\lstlistingname}{Codeauszug}
%\lstset{
%	language=Java,
%	numbers=left,
%	columns=fullflexible,
%	aboveskip=5pt,
%	belowskip=10pt,
%	basicstyle=\small\ttfamily,
%	backgroundcolor=\color{black!5},
%	commentstyle=\color{darkgreen},
%	keywordstyle=\color{blue},
%	stringstyle=\color{gray},
%	showspaces=false,
%	showstringspaces=false,
%	showtabs=false,
%	xleftmargin=16pt,
%	xrightmargin=0pt,
%	framesep=5pt,
%	framerule=3pt,
%	frame=leftline,
%	rulecolor=\color{green},
%	tabsize=2,
%	breaklines=true,
%	breakatwhitespace=true,
%	prebreak={\mbox{$\hookleftarrow$}}
%}

\lstdefinelanguage{CSS}{
	keywords={color,background-image:,margin,padding,font,weight,display,position,top,left,right,bottom,list,style,border,size,white,space,min,width, transition:, transform:, transition-property, transition-duration, transition-timing-function},	
	sensitive=true,
	morecomment=[l]{//},
	morecomment=[s]{/*}{*/},
	morestring=[b]',
	morestring=[b]",
	alsoletter={:},
	alsodigit={-}
}

% JavaScript
\lstdefinelanguage{JavaScript}{
	morekeywords={typeof, new, true, false, catch, function, return, null, catch, switch, var, if, in, while, do, else, case, break},
	morecomment=[s]{/*}{*/},
	morecomment=[l]//,
	morestring=[b]",
	morestring=[b]'
}

\lstdefinelanguage{HTML5}{
	language=html,
	sensitive=true,	
	alsoletter={<>=-},	
	morecomment=[s]{<!-}{-->},
	tag=[s],
	otherkeywords={
		% General
		>,
		% Standard tags
		<!DOCTYPE,
		</html, <html, <head, <title, </title, <style, </style, <link, </head, <meta, />,
		% body
		</body, <body,
		% Divs
		</div, <div, </div>, 
		% Paragraphs
		</p, <p, </p>,
		% scripts
		</script, <script,
		% More tags...
		<canvas, /canvas>, <svg, <rect, <animateTransform, </rect>, </svg>, <video, <source, <iframe, </iframe>, </video>, <image, </image>, <header, </header, <article, </article
	},
	ndkeywords={
		% General
		=,
		% HTML attributes
		charset=, src=, id=, width=, height=, style=, type=, rel=, href=,
		% SVG attributes
		fill=, attributeName=, begin=, dur=, from=, to=, poster=, controls=, x=, y=, repeatCount=, xlink:href=,
		% properties
		margin:, padding:, background-image:, border:, top:, left:, position:, width:, height:, margin-top:, margin-bottom:, font-size:, line-height:,
		% CSS3 properties
		transform:, -moz-transform:, -webkit-transform:,
		animation:, -webkit-animation:,
		transition:,  transition-duration:, transition-property:, transition-timing-function:,
	}
}

\lstdefinestyle{htmlcssjs} {%
	% General design
	%  backgroundcolor=\color{editorGray},
	basicstyle={\footnotesize\ttfamily},   
	frame=b,
	% line-numbers
	xleftmargin={0.75cm},
	numbers=left,
	stepnumber=1,
	firstnumber=1,
	numberfirstline=true,	
	% Code design
	identifierstyle=\color{black},
	keywordstyle=\color{blue}\bfseries,
	ndkeywordstyle=\color{editorGreen}\bfseries,
	stringstyle=\color{editorOcher}\ttfamily,
	commentstyle=\color{brown}\ttfamily,
	% Code
	language=HTML5,
	alsolanguage=JavaScript,
	alsodigit={.:;},	
	tabsize=2,
	showtabs=false,
	showspaces=false,
	showstringspaces=false,
	extendedchars=true,
	breaklines=true,
	% German umlauts
	literate=%
	{Ö}{{\"O}}1
	{Ä}{{\"A}}1
	{Ü}{{\"U}}1
	{ß}{{\ss}}1
	{ü}{{\"u}}1
	{ä}{{\"a}}1
	{ö}{{\"o}}1
}

%Anpassungen für das Abkürzungsverzeichnis
\newglossarystyle{dottedlocations}{%
	\renewcommand*{\glossaryentryfield}[5]{%
		\item[\glsentryitem{##1}\glstarget{##1}{##2}] \emph{##3}%
		\unskip\leaders\hbox to 2.9mm{\hss.}\hfill##5}%
	\renewcommand*{\glsgroupskip}{}%
}

%Titelformen - gewünschtes Layout einkommentieren

%%Graduation
\makeatletter

\newcommand*{\gradeType}[1]{\gdef\@gradeType{#1}}
\newcommand*{\firstExaminer}[1]{\gdef\@firstExaminer{#1}}
\newcommand*{\secondExaminer}[1]{\gdef\@secondExaminer{#1}}
\newcommand*{\matrikelnr}[1]{\gdef\@matrikelnr{#1}}
\newcommand*{\submitDate}[1]{\gdef\@submitDate{#1}}

\renewcommand*{\maketitle}{
	\begin{titlepage}
		\newgeometry{left=2.5cm,right=2.5cm,top=2.5cm,bottom=2.5cm}
		\begin{figure}[H]
			\centering
			\includegraphics[]{resources/images/S04_HTW_Berlin_Logo_pos_FARBIG_RGB}
			\label{img:htw_logo}
		\end{figure}
		\begin{center}
			\rule{\textwidth}{0.2mm}
			{\Large \bfseries \textcolor{htw_green}{\@title}\par}
			\rule{\textwidth}{0.2mm}
			\vskip 0.2cm
			{\large \bfseries Masterarbeit\par}
			\vskip 2cm
			{\large Name des Studiengangs}
			\vskip 0.2cm
			{\large \bfseries Internationale Medieninformatik}
			\vskip 0.2cm
			{\large \bfseries \textcolor{htw_green}{Fachbereich 4}}
			\vskip 0.2cm
			{\large vorgelegt von}
			\vskip 0.2cm
			{\large \bfseries Moritz Thomas}
			\vskip 0.2cm
			{\large \@matrikelnr}
			\vskip 2cm
			{\large Datum:}
			\vskip 0.2cm
			{\large Berlin, 01.09.2020}
			\vskip 0.5cm
			{\large \bfseries Erstgutachter\_in: \@firstExaminer}
			\vskip 0.2cm
			{\large \bfseries Zweitgutachter\_in: \@secondExaminer}
		\end{center}
		\restoregeometry
	\end{titlepage}
}
\makeatother
\gradeType{Master of Science (M.Sc.)}
\secondExaminer{Sumit Kapoor}

%Research paper
%\include{titles/research_papger}
%\subTitle{Ein optionaler Untertitel der Arbeit}
%\researchPart{A}

%Angaben zur Arbeit und dem Author (von beiden Layouts genutzt)
\title{Untersuchung der Algorithmen und Prozesse der Standortanalyse im Kontext einer Filialplanung eines Einzelhändlers}
\author{Moritz Thomas}
\matrikelnr{s0544877}
\submitDate{25.04.2017}
\firstExaminer{Prof. Dr. Tobias Lenz}

%Verzeichnisse generieren
\makeglossaries
\loadglsentries{references/glossary_acronyms.tex}
\setacronymstyle{long-short}

\makeindex[columns=2, title=Stichwortverzeichnis, options= -s resources/styles/indexstyle.ist, intoc]
\indexsetup{level=\chapter*,toclevel=chapter}

%Start des Inhalts
\begin{document}

%Notwendiger Workaround
\pagenumbering{alph}

%Deckblatt erzeugen
\maketitle

\pagenumbering{Roman}

% Eigenständigkeitserklärung
\addchap{Eigenständigkeitserklärung}

Hiermit versichere ich, dass ich die vorliegende Masterarbeit selbstständig und nur unter
Verwendung der angegebenen Quellen und Hilfsmittel verfasst habe. Die Arbeit wurde bisher
in gleicher oder ähnlicher Form keiner anderen Prüfungsbehörde vorgelegt.

\vskip 1cm

Berlin, den 01.03.2021

\begin{figure}[ht]
	\includegraphics[scale=0.5]{resources/images/Unterschrift Kopie.jpg}
\end{figure}

Moritz Thomas \clearpage
%\include{chapter/Vorwort} \clearpage
\chapter*{Abstract}
Die Besuchswahrscheinlichkeit eines Kunden hat großen Einfluss auf den Umsatz und Marktanteil einer Filiale.
Großen Verkaufsketten des Lebensmitteleinzelhandels stehen hierzu in ständigem Konkurrenzkampf.
Neben der eigene Preispolitik oder individuellem Sortiment, kann vor Allem über die Eröffnung einer weiteren Filiale Einfluss auf Die Besuchswahrscheinlichkeit genommen werden.
Bei der Wahl des Standortes sollten mehrere Faktoren in Betracht gezogen werden.
Die Berechnung der Wahrscheinlichkeit kann über Gravitationsmodelle  erfolgen.\\
Eines der bekanntesten Modelle wurde von David L. Huff in 1962 entwickelt.
Es hat den Zahn der Zeit überstanden und ist nach wie vor aufgrund der einfachen Anwendung und dennoch großen Aussagekraft sehr angesehen und hat viele Erweiterungen hervorgebracht.
Für die digitale Anwendung als Komponente eines Web-Geoinformationssystems (Web-GIS) wird dieses Modell mit moderner Technologie, in Form von Angular und OpenLayers, prototypisch implementiert und deren Anwendung bewertet.

The probability of a customer visiting a store has a major influence on its sales and market share.
Large food retail chains are in constant competition with each other in this respect.
In addition to the company's own pricing policy or individual product range, the likelihood of a customer visiting a store can be influenced above all by the opening of a further store.
When choosing a location, several factors should be taken into account.
The calculation of the probability can be done by gravity models.
One of the best known models was developed by David L. Huff in 1962.
It has survived the ravages of time and is still highly regarded for its ease of use yet great predictive power, and has spawned many extensions.
For digital application as a component of a web geographic information system (web GIS), this model is prototyped using modern technology, in the form of Angular and OpenLayers, and its application is evaluated.
 \clearpage

%Inhaltsverzeichnis
\tableofcontents \newpage

%Hauptteil
\pagenumbering{arabic}
\chapter{Einleitung}
\section{Einführung}
Im Jahr 2019 gaben private Haushalte in Deutschland rund 197,3 Milliarden Euro für Lebensmittel aus. \cite{statista2019Ausgaben}
Dies verteilt sich auf rund 34.947 Geschäfte zwar kontrollieren die vier Unternehmen Edeka, Rewe, Aldi und die Schwarz-Gruppe rund 70 Prozent des Marktes, dennoch herrscht ein Konkurrenzkampf der jeden noch so kleinen Vorteil gegenüber den Wettbewerbern in direkten Umsatz, Marktanteil oder Kundenzufriedenheit Anstieg nieder spiegelt.  \cite{statista2018Geschäfte}
Das Spielfeld ist hierbei vielfältig und wechselhaft, Faktoren ändern sich. In Zeiten der fortgeschrittenen Digitalisierung sind Unternehmen mittlerweile auf technische Unterstützung angewiesen, um nicht abgehängt zu werden. Sämtliche Bereiche der Maschinerie Lebensmitteleinzelhandel (im Folgenden LEH) sind teilweise vollständig oder zu großen Teilen von technischen Prozessen durchzogen.
Dennoch gibt es LEH bereits seit Jahrzehnten und gängige Prozesse haben sich etabliert und verbreitet. So auch die Standortanalyse und die Filialplanung. Eine spannende Aufgabe der Unternehmen und der Wissenschaft ist es nun neue Technologien für die digitale Umsetzung der gegebenen Prozesse zu nutzen und durch das Zusammenspiel Optimierung und Innovation zu erreichen. 

\section{Motivation}
Viele der in der Standortanalyse verwendeten Prozesse und Algorithmen sind geo-mathematischer oder geo-informatischer Natur und liegen teils komplexer Berechnungen zu Grunde.
Diese zu erkunden und ergründen haben sich eigene Wissenschaftsbereiche aus der Mathematik, Betriebswirtschaftslehre, Informatik und dem Ingenieurswesen gebildet, die diverse Studiengänge und Ausbildungen beheimaten.
Um diese komplexen Themen zu vereinfachen und den Mitarbeitern des LEH die tägliche Arbeit zu erleichtern, können digitale Prozesse eingesetzt werden.
In benutzerfreundlichen, einfach zu verstehenden und visuell ansprechenden Anwendungen sollen sich die Algorithmen und Prozesse der Standortanalyse und Filialplanung verbergen. 
Solche Anwendungen werden Geo-Informationssysteme (kurz und im Folgenden GIS) genannt.
Auf dem Markt existieren derer bereits einige.
Big Player sind zum Beispiel Pitney Bowes, ESRI, oder Autodesk.
Ebenso existieren einige OpenSource Angebote wie zum Beispiel GRASS GIS oder QGIS.
In der Praxis benötigen die Unternehmen dennoch oft Individuallösungen, die entweder auf bestehender Software aufbauen oder diese integrieren.
Ziel dieser Arbeit ist es daher eine Anwendung zu entwickeln, die Algorithmen und Prozesse der Standortanalyse und Filialplanung einfach implementiert und in eine solche Individuallösung integriert werden kann.
Zu dieser Anwendung gehören also lediglich eine Karte, Karten-Werkzeuge und Geo-Objekte als Teilobjekte eines kompletten GIS.


\section{Abgrenzung}
In dieser Arbeit werden Themenfelder des Geo-Marketings, der Geo-Informatik und -Mathematik sowie diverse Technologien behandelt. 
Die resultierende Anwendung soll nichts weiter als ein Prototyp darstellen und ist keinesfalls ein komplettes GIS. 
Vielmehr wird sich auf die Ausarbeitung des Huff-Models zu Standortanalyse und Filialplanung konzentriert mit dem Fokus auf Umsetzbarkeit innerhalb einer Web-Anwendung mit OpenSource Technologien.  \clearpage
\chapter{Grundlagen}
Zu den Grundlagen dieser Arbeit zählen Theorien und Konzepte aus dem Geomarketing und der Geoinformatik sowie die angewendeten Technologien im Prototyp.
Die folgenden Kapitel stellen die wichtigsten Informationen bereit, die ein allgemeines Verständnis der Anwendung ermöglichen.

\section{Geomarketing}
Aus dem Handbuch Geomarketing von Michael Herter: 
Geomarketing analysiert aktuelle wie potenzielle Märkte nach räumlichen Strukturen, um den Absatz von Produkten effektiver planen und messbar steuern zu können \footcite{herter_handbuch_2018}.
Ergänzend befasst sich Geomarketing mit der Beschreibung, Analyse und dem Vergleich beliebiger Märkte und Standorte hinsichtlich ökonomischer Charakteristiken und Potenziale durch Referenzierung und flächendeckende Berechnung von Marktdaten auf geographische Strukturen \footcite{geomarketing_def}.
Kurz gesagt beschreibt Geomarketing also sämtliche Aspekte des Marketings, die geografischen Bezug haben.

Zu relevanten Themenbereichen des Geomarketings für diese Arbeit gehören Standortanalyse /-planung, Filialplanung sowie die Gravitationsanalyse und das Konzept des Gravitationsmodels.
Weiterhin werden die Begriffe Geodaten und Marktdaten erläutert.

\subsection{Standortanalyse}
Allgemein erörtert die Standortanalyse Beschreibung, Untersuchung und Unterscheidung von guten und schlechten IST-Standorten sowie potenziellen neuen Standorten und deren Umfeld hinsichtlich der Eignung für den Absatz bestimmter Produkte \footcite{geomarketing_standortanalyse}.

Für die Betrachtung von potenziellen neuen Standorten werden unternehmensinterne Daten (falls vorhanden) und externe Daten anhand von verschiedenen Standortfaktoren untersucht, mit dem Ziel sämtliche potenzielle Standorte auf möglichst wenig Alternativen zu begrenzen und schließlich analytisch die Beste zu bestimmen. 
Die Unterscheidung kann hierbei in höchster Ebene in vier Kategorien erfolgen \footcite{haas_standortfaktoren}:

\textbf{\emph{Zugehörigkeit zur Leistungserstellung}} beinhaltet Faktoren zur Beschaffung, Produktion und zum Absatz.

\textbf{\emph{Grad der monetären Quantifizierbarkeit}} beinhaltet harte und weiche Standortfaktoren. 
Harte Standortfaktoren sind immer quantifizierbar und dienen daher als Grundlage für wirtschaftliche Berechnungen und Kosten.
Der Einfluss weicher Standortfaktoren kann nicht eindeutig bestimmt werden und kann mit der selektiven Clusterung all der Faktoren beschrieben werden, die auf dem individuellen Raumempfinden der Menschen in ihrer Lebens- und Arbeitswelt basieren.

\textbf{\emph{Maßstabsebene}} beschreibt Faktoren der Makro-, Meso- und Mikroebene oder anders der Länder-, Region- und Gemeindeebene.

\textbf{\emph{Grad der Spezifität}} beschreibt Sektor- und Branchenspezifische Faktoren.

Die Abbildung \ref{img:standortfaktoren} bietet hierzu eine detaillierte Auflistung der Kategorien mit Beispielen.

\begin{figure}[H]
	\centering
	\includegraphics[scale=0.2]{resources/images/Standortfaktoren.png}
	\caption{Eigene Darstellung Standortfaktoren}
	\title{Standortfaktoren}
	\label{img:standortfaktoren}
\end{figure}

\newpage
Weitere zusätzlich relevante Aspekte können sein:
\begin{itemize}
	\item Zielgruppe - Wer sind die Kunden? Ist das Verkaufsmodell B2C oder B2B?
	\item Demografische Merkmale - Alter, Geschlecht und Wohnort der Kunden
	\item Sozioökonomische Betrachtung - Gibt es einen Zusammenhang zwischen Beruf, Bildungsstand oder Einkommen und dem Kaufverhalten?
	\item Psychografische Merkmale - Hat der Lebensstil, Werte, Motivation oder Ähnliches Einfluss auf das Kaufverhalten?
	\item Wettbewerbsdichte
	\item Preispolitik der Wettbewerber
	\item Größe des Einzugsgebiets
	\item Kaufkraftwerte des Gebiets
	\item Erreichbarkeit
	\item Mietpreise
	\item Attraktivität des Standortes
\end{itemize}

Obwohl Filialen geographisch sowie aus betriebswirtschaftlicher Sicht Unternehmensstandorte sind, kann zwischen Filialplanung und Standortplanung unterschieden werden und dementsprechend werden andere Faktoren bei der Standortanalyse wichtiger oder hinzugezogen.
Für die Filialplanung zusätzlich relevant werden folgende Faktoren:

\begin{itemize}
	\item Bekanntheitsgrad, Markenstärke - Was zeichnet die eigene Marke aus? Was setzt sie ab?
	\item Freies Potential - Herrscht genug Nachfrage?
	\item Kannibalisierungs-Effekte - Ist der Standort zu nah an anderen, bestehenden Filialen?
	\item Logistikkosten - Kann ich Lieferwege und Lagerstandorte optimal nutzen?
\end{itemize}

Für eine einfache mathematische Standortanalyse können nun zunächst die Kategorien anhand der eigenen Marke als mehr oder weniger relevant bewertet und gewichtet werden.
Außerdem müssen die einzelnen Kategorien für jeden Standort bewertet werden. 
Die Bewertungen der Kategorien müssen nun mit der Gewichtung der Kategorie multipliziert werden, um die genaue Faktorbewertung des Standorts für jede Kategorie zu erhalten. 
Die Faktorbewertungen summiert ergeben nun Standortbewertung. 
Die Standortbewertungen gegenübergestellt stellen die Standortanalyse abschließend gelistet dar.

\subsection{Standortplanung/ Filialplanung}
Bereits im vorherigen Kapitel erwähnt, wird zwischen der Standortplanung und der Filialplanung geringfügig unterschieden.
So legt die Filialplanung ihren Fokus auf mehrere Standorte in einem Filialnetz zusammengefasst, welches bei der Analyse berücksichtigt werden muss.
Wohingegen die Standortplanung sich eher auf einige wenige Standorte konzentriert.

Beide Methoden verwenden jedoch in der vorher durchgeführten Standortanalyse herausgestellte Erkenntnisse, um die potenziellen Standorte einzuordnen und eine Entscheidung der nächsten Filialeröffnung zu treffen.

\subsection{Geodaten}
Geodaten sind strukturierte codierte Angaben zur quantitativen und qualitativen Beschreibung von natürlichen oder definierten Objekten der realen Welt. 
Geodaten vermessen die Welt und beschreiben geographische (Teil-)räume und Orte \footcite{geomarketing_geodaten}.

Geodaten werden in mehrere Typen klassifiziert und unterliegen definierten Standards, die sich am Markt durchgesetzt haben \footcite{gistandards.eu}.

So definiert die ISO 19115 einen Standard zur Beschreibung geographischer Informationen anhand von Metadaten \footcite{19115_iso}.
Wichtigste standardisierte Dienste sind unter Anderem:

\begin{itemize}
	\item WMS - Web Map Service zum Teilen von Karten
	\item WFS - Web Feature Service zum Teilen von Feature-Daten
\end{itemize}

Und zu den wichtigsten standardisierten Datenformaten zählen:

\begin{itemize}
	\item GeoJSON - JSON Format mit Geo-Spezifika
	\item KML - Keyhole Markup Language 
	\item GPKG - GeoPackage definiert vom Open Geospatial Consortium
\end{itemize}

Im Prototyp verwendete Formate sind GeoJSON für Features sowie einfache Bilder (PNG) der Hintergrundkarte.

\subsection{Marktdaten}
Marktdaten oder Marktinformationen beschreiben die individuelle regionale Charakteristik geographischer Gebiete oder Standorte mittels qualitativer Merkmale \footcite{geomarketing_marktdaten}.

Es gibt unternehmensinterne sowie externe Marktdaten.
Marktdaten können aus einem internen CRM-System stammen und über Umfragen, Marktforschung, Erhebungen als Primärdaten selbst erhoben werden oder aus staatlichen Statistiken und Branchen- und Wirtschaftsverbänden als Sekundärdaten eingeholt werden.

Im Prototyp verwendete Marktdaten sind aus Sekundärdaten erstellte Beispieldaten für Filialen und Gebiete.

\section{Geoinformationssysteme}
\label{sec:gis}
Geoinformationssysteme, kurz GIS, sind Informationssysteme zur Erfassung, Bearbeitung, Organisation, Analyse und Präsentation räumlicher Daten \footcite{geomarketing_gis}.  
Ähnlich anderer Informationssysteme besteht ein GIS aus Hardware (Computer, Server, Drucker etc.), Software mit Analyse-Tools (Zeichnen, Kalkulation) und räumlichen Daten (Koordinatensystem, Karten, Geometrien etc.). 
Gegebenenfalls wird die Liste um eine Verwaltungsebene ergänzt sobald die Daten und Funktionen des GIS Rollen und Rechte behaftet sind.
Erste GIS Systeme stammen aus den sechziger Jahren (Canada Geographic Information System \footcite{esri_cgis}). 
Zu den ersten Nutzern der Systeme gehörten vor Allem Behörden und Universitäten, so wurden viele grundlegende theoretische Konzepte an der Harvard University von Professor Howard Fisher aufgestellt \footcite{gis_harvard_2005}.
Mittlerweile haben sich viele Web-GIS etabliert, hierbei wird das Informationssystem über eine Website veröffentlicht oder benutzt. 
Zu den größten Vertretern moderne Web-GIS zählen vor Allem Google Maps, Bing Maps, OpenStreetMaps als OpenSource-Alternative oder etwas kleinere Anbieter wie HERE oder Yandex.Maps.
Kommerzielle Web-Gis bieten meist ein eingeschränktes Funktions-Set, was sie nicht als Produkt für individuelle Software-Lösungen für Firmen in Frage kommen lässt daher greifen viele Firmen auf kommerzielle Desktop-GIS zurück oder lassen sich ganz individuelle Systeme bauen, die aus Web- und Desktop-GIS bestehen.
Zu den bekanntesten GIS zählen Produkte von ESRI, Autodesk, Pitney Bowes oder CAIGOS. 

In der Geophysik wird die Erde nicht als Kugel sondern Ellipsoid bezeichnet \footcite{jung_figur_1956}.
Wenn man jedoch eine Karte elektronisch auf einem Bildschirm betrachtet, dann sieht man ein Rechteck und keine Anzeichen der Krümmung eines Globus.
Die Problematik der Projektion der Erde auf eine flache Darstellung hat über den Lauf der Jahre mehrere Lösungen produziert, die jedoch alle mit Koordinatensystemen und Projektionsrechnungen zu tun haben, und bietet Stoff für ein eigenes Wissenschaftsfeld.
Aufgrund dessen versuchen die nächsten Zeilen, die Problematik und Anwendung im Prototyp kurz zu beschreiben.

In einem GIS muss die Projektion zwangsläufig enthalten sein und das bestenfalls für mehrere Projektionstypen.
Die meisten Web-Karten wie Google Maps oder Bing Maps benutzen die Projektion World Geodetic System 1984 (kurz WGS 84) mit dem Koordinatenreferenzsystem (engl. CRS) der European Petroleum Survey Group (kurz EPSG) 4326, welches Koordinaten in Grad darstellt \footcite{epsg.io_4326}.
OpenLayers hingegen benutzt standardmäßig WGS 84 mit EPSG 3857, welches Koordinaten in Metern zur Ursprungskoordinate darstellt \footcite{epsg.io_3857}.
Die Abbildung \ref{img:epsg.io} zeigt die Koordinaten eines Punktes in Berlin in beiden Koordinatenreferenzsystemen.

\begin{figure}[H]
	\centering
	\includegraphics[scale=0.6]{resources/images/projection_epsg.png}
	\caption{Bildschirmaufnahme der Transformation  von EPSG 4326 zu EPSG 3857 von epsg.io \footcite{epsg.io_transform}}
	\label{img:epsg.io}
\end{figure}

Im wesentlichen werden im Prototypen die oben beschriebenen Projektionen verwendet.
Die Geodaten der Filialen sowie der Zensusgebiete sind in EPSG 4326 erfasst und müssen dementsprechend vor der Darstellung transformiert werden.
Entsprechende Funktionen sind im Kapitel \ref{ch:implementierung} Implementierung beschrieben.

\section{Genutzte Technologie}
Für die prototypische Ausarbeitung des Modells wurde auf modernste Technologien und Frameworks zugegriffen. So wird als Karten-Framework /gls(ol) in der Version 6.4.3 verwendet sowie /gls(angular) in der Version 10.
Die Darstellung \ref{img:tech_stack} zeigt die genutzten Technologien als Grafik.

\begin{figure}[H]
	\centering
	\includegraphics[]{resources/images/tech_stack.png}
	\caption{Eigene Abbildung des Technologie Stacks der Anwendung}
	\label{img:tech_stack}
\end{figure}

OpenLayers bietet ein breites Portfolio an Funktionen und Features passend für sämtliche Anforderungen realer Geo-Informationssysteme die in der Praxis verwendet werden. Im Bezug auf die prototypische Anwendung dieser Arbeit sind vor Allem die Unterstützung der Darstellung von Punkten (Filialen) und der Gravitations-Gebiete in GeoJSON-Format auf verschiedenen Layern notwendig sowie die einfache Konfigurierbarkeit von Karten-Interaktion wie zum Beispiel das Platzieren, Verschieben und Editieren von Punkt-Objekten (Filialen) auf der Karte.

Weiterführend soll die Anwendung schnell, kompakt und modern sein, um die Relevanz aus Performance Gründen auf dem Markt gewährleisten zu können. Daher erfolgt die Umsetzung mit dem auf TypeScript basierenden Framework Angular. Besonderes Augenmerk liegt hierbei im Prototypen eigentlich nur auf der simplen, kompakten und schnellen Auslieferung eines Web-Servers als Host der Anwendung. Zu den für die weiterführende Entwicklung relevant werdenden Features des Frameworks zählen eine große Community, einen fortlaufenden Support sowie eine fortlaufende Entwicklung durch Google, die Verwendung von TypeScript einem Superset von JavaScript mit Verbesserter Funktionalität und sämtlichen Features, die für Entwicklung einer effizienten und anspruchsvollen Single-Page-Webanwendung benötigt werden.  \clearpage
\chapter{Konzept}
Die Algorithmen und Prozesse einer Filialplanung spiegeln sehr gut einen allgemeinen Anwendungsfall moderner GIS. 
Neben einer einfachen Hintergrundkarte sind die Mitarbeiter auf Zeichenwerkzeuge, Geo-Daten-Anzeige, thematische Karten und Kennzahlen Berechnungen angewiesen, um akkurate und fundierte Prognosen und Planungen treffen zu können.
Viele theoretische Konzepte und Prozesse passieren hierbei im Hintergrund und sind nicht direkt ersichtlich.
In den folgenden Kapiteln werden die wichtigsten Beschrieben mit dem Fokus auf den Algorithmus der hinter dem Gravitations-Model von Huff steht.
\section{Algorithmen und Prozesse}
Um eine realistische Abbildung der Welt auf eine Karte zu bringen benötigt es eine Umrechnung eines Ovales, quasi der Erdkugel, auf eine Rechteck, im Fall eines Web-Gis des Bildschirms. 
\subsection{Gravitations-Modelle}
Das \textit{Huff-Modell} (engl. Huff Gravitation Model) ist ein mathematisches Modell zur Abgrenzung und Segmentierung von Marktgebieten \cite{Roy2004}.
Das Modell bestimmt die Wahrscheinlichkeit, mit der Kunden einen Standort (Filiale, Einkaufszentrum) in Abhängigkeit von Distanz und Attraktivität aufsuchen. 
Die Wahrscheinlichkeiten werden nun auf der Karte als Punkte dargestellt, die rund um den Standort platziert werden. 
Die Punkte der selben Wahrscheinlichkeiten werden zu Isowahrscheinlichkeitslinien verbunden. 
So entstehen zu jedem Standort verschiedene Gravitationsebenen, die farblich gekennzeichnet werden. 
Die Beeinflussung der verschiedenen Gravitationsebenen führt zu mehreren Farbverläufen, die ein komplexes Bild der Standort Landschaft bilden.
In Seiner einfachsten Form berücksichtigt das Modell nur die Distanz und eine simple Kennzahl der Attraktivität (zum Beispiel in Form eines Rankings) für die Wahrscheinlichkeitsberechnung. 
Für die komplexere Berechnung mit mehr Faktoren wird das MCI-Modell von Nakanishi und Cooper verwendet, welches die Weiterentwicklung der einfachen Huff-Modells darstellt \cite{mciModell}.

Das \textit{MCI-Modell} ...
\section{Architektur}
 \clearpage
\chapter{Implementierung}
Das Kapitel Implementierung erfasst die technische Dokumentation des erarbeiteten Prototyps.
Im Detail werden in den folgenden Seiten die technische Umsetzung des in Kapitel \ref{ch:concept} entworfenen Konzepts beschrieben. 
Beschriebene Funktionalitäten werden mit Bildausschnitten unterstützt.

\section{Prototyp}
Die Anwendung wurde als Angular Projekt mittels der Angular CLI erstellt. 
Über den CLI Befehl 

\begin{lstlisting}[language=bash]
$ ng new gravitationsmodel
\end{lstlisting}

generiert die CLI ein kompilierbares und ausführbares Angular Projekt mit essentiellen Abhängigkeiten und Strukturen.
Angular verwendet standardmäßig NPM als package manager.
Eine \emph{package.json}, welche sämtliche Abhängigkeiten dokumentiert, wird bereits mit erstellt.
Um das Setup abzuschließen müssen weitere Abhängigkeiten in Form von \emph{npm packages} installiert werden.
Über den Befehl 

\begin{lstlisting}[language=bash]
$ ng add ol
\end{lstlisting}

fügt die CLI automatisch das Paket von OpenLayers hinzu und für die korrekte Typisierung in TypeScript das notwendige \emph{types} Paket für OpenLayers hinzu.

Nachdem die technischen Voraussetzungen geschaffen sind kann mit der Implementierung begonnen werden.
Auch hierbei bietet die CLI Unterstützung in Form von \emph{Scaffolding} Befehlen.

\begin{lstlisting}[language=bash]
$ ng generate <schematic> [name]
\end{lstlisting}

generiert Komponenten, Services, Models, Klassen und weitere Code-Gerüste durch den passenden Präfix-Parameter und Namen.
Angular Komponenten bestehen immer aus einer TypeScript Klasse (\emph{.component.ts}), einem Template (\emph{.component.html}) sowie Dateien für Styling (\emph{.css} oder \emph{.scss}) und Test (\emph{.component.spec.ts}).

\begin{lstlisting}[language=bash]
$ ng generate component objektfenster
\end{lstlisting}

erstellt \emph{objektfenster.component.ts}, \emph{objektfenster.component.html} sowie \emph{objektfenster.scss} und \emph{objektfenster.component.spec.ts} Dateien.

Mit Hilfe der CLI lassen sich so die Code-Gerüste schnell erstellen und die CLI übernimmt sogar die in Angular nötige \emph{Dependency Injection} indem neue Komponenten direkt in der entsprechenden \emph{.module.ts} deklariert werden \footcite{angular_cli}.

Nachdem die Komponenten erstellt wurden kann nun mit der Implementierung der fachlichen Funktionen begonnen werden.
Der Kern der Anwendung ist eine Karte mit der interagiert wird.
Um diese darzustellen, muss zunächst ein OpenLayers Map-Objekt erstellt und in den DOM eingehängt werden \footcite{openlayers_map}.
Dies geschieht in der OpenLayersMap-Komponente und dem BaseMap-Service.
Über das bereits installierte OpenLayers Paket wird das Map-Objekt importiert und initialisiert.

\begin{lstlisting}[language=JavaScript]
	import Map from 'ol/Map';
	import {fromLonLat} from 'ol/proj';
	import {View} from 'ol';

	const targetId = 'map';
	const longitude = 13.451338;
	const latitude = 52.503707;
	const zoomLevel = 11;
	
	const coordinate = fromLonLat([longitude, latitude]);
	
	const view = new View({
		center: coordinate,
		zoom: zoomLevel
	});
	
	new Map({
		target: targetId,
		layers: [],
		controls: [],
		view,
	});


\end{lstlisting}

Die Karte wird mit der notwendigen Id des HTML-Elements, in welchem die Karte angezeigt werden soll, und einem View initialisiert.
Das View-Objekt besteht aus einer Center-Koordinate und der initialen Zoomstufe und stellt den Begrenzungsrahmen (engl. Bounding Box) der Karte dar.
Zu beachten ist ebenfalls hierbei die Transformation der Koordinaten von EPSG:4326 in das in OpenLayers standardmäßig genutzte EPSG:3857 über die Funktion \emph{fromLonLat} \clearpage
\chapter{Auswertung}
Im letzten Teil dieser Arbeit wird das Huff-Modell und dessen Umsetzung im Prototypen ausgewertet.\\
Was sind die Anwendungsfälle des Huff-Modells?
Wozu kann der Prototyp verwendet werden?
Ist der Prototyp performant?
Ist das Huff-Modell eine gute Wahl für einen Algorithmus der Filialplanung und Standortanalyse?
Wo stößt das Modell an seine Grenzen?
Was kann verbessert, erweitert oder ersetzt werden?

Unter Anderem diesen Fragen wird mit Ausblick auf die Zukunft kritisch Antwort gegeben bevor ein finales Fazit gezogen wird.

\section{Bewertung}
\label{sec:bewertung}
Bei der Bewertung des Prototypen muss einerseits die Umsetzung des Huff-Modells in einer Webkarte mit modernen Technologien und andererseits das Modell selber bewertet werden.
Hierzu werden zunächst Anwendungsfälle des Modells beschrieben, die Verwendung des Prototypen unter Einbeziehung der Performance für diese erklärt und Verbesserungen und Erweiterungen vorgestellt.
Danach wird das Modell selber in Frage gestellt und mögliche Alternativen oder Erweiterungen vorgestellt.

Obwohl das Huff-Modell kein neues Phänomen ist und schon mehrere Jahrzehnte alt ist, hat es sich als verlässlich erwiesen und findet nach wie vor Gebrauch.
Die liegt vor Allem an der einfachen und schnellen Verwendung. 
Bereits mit wenig Informationen können simple Prognosen zu Besucherwahrscheinlichkeit erstellt werden.\\
Zu den Hauptanwendungsfällen zählen:

\begin{itemize}
	\item Filialplanung
	\item Expansionsplanung
	\item Verteilgebietsplanung
	\item Kundenansprache
\end{itemize}

Die Filialplanung und Expansionsplanung sowie die Verteilgebietsplanung und die Kundenansprache sind ähnlich zu betrachten, da sie sich nur geringfügig Unterscheiden.\\
Der Prototyp kann alle aufgelisteten Fälle grundlegend abdecken,könnte aber aber bei gezielten Erweiterungen eine noch bessere Unterstützung für die spezifischen Anforderungen bieten.
So lassen sich Besuchswahrscheinlichkeiten in bestehendem Netz sowie mit neuen Filialen berechnen und vergleichen.
Potenzielle neue Standorte können dem bestehenden Netz hinzugefügt werden und deren Auswirkung auf Wahrscheinlichkeiten und Umsatz werden direkt visuell und numerisch sichtbar, was bei direktem Vergleich eine Erstellung einer Reihenfolge anhand von Umsatzgewinn oder -verlust und Marktanteil der potenziellen neuen Standorte erlaubt.\\ 

Ebenso kann sehr leicht ein Verteilgebiet für Handzettel (Werbung) abgegrenzt werden, dass sich auch wirklich nur an Kunden richtet, die wahrscheinlich in der Filiale einlaufen würden.
Außerdem sind aus den Gebieten des Verteilgebiets erste Informationen zu Kaufkraft, Einwohneranzahl und Ausgaben ersichtlich, welches eine persönlichere Ansprache der Kunden zulässt, die auf demografischen Analysen beruht.\\
Eine sinnvolle Erweiterung für die Verteilgebietserstellung wäre die Gruppierung mehrerer Gebiete zu einem neuen Verteilgebiet in einem neuen Layer.
Das neue Gebiet könnte dann eine eigene Farbe erhalten und nach Bedarf ein- und ausgeblendet werden oder in der Z-Ebene nach oben oder unten verschoben werden.
Für das neue Gebiet können dann die Daten der eingeschlossenen Gebiete summiert werden und für das potenzielle Verteilgebiet angezeigt werden.

Die Berechnung des Modells ist bereits angemessen performant mit initialer Lade- und Renderzeit von unter 1 Sekunde für die Gebiete und Filialen sowie Berechnungs- und Rerenderzeit der Gebiete von ebenfalls unter 1 Sekunde.
Durch Optimierung beim Laden und Rendern durch zum Beispiel explizite Nutzung von Cache-Objekten für Features und Styles oder Überarbeitung des Codes für die Berechnung des Modells, könnte hier eventuell noch ein wenig mehr Performance optimiert werden.

Der Prototyp kann das Huff-Modell sehr gut darstellen und bietet echten Mehrwert in Prozessen des Geomarketings.
Dennoch hat das Modell auch einige Schwächen oder Ungenauigkeiten, wie unter Anderem bereits im Artikel "Application of network Huff model for commercial network planning at suburban – Taking Wujin district, Changzhou as a case" von Haozhi Pan,Yongfu Li und Anrong Dang beschrieben \footcite{pan_application_2013}.
Die Autoren stellen vor Allem die Bestimmung der Parameter $\alpha$ und $\beta$ sowie die Berechnung der Distanzvariable D als problematisch hervor.
Während es für die Bestimmung der Parameter meist an regionalen Marktdaten mangelt oder eine Regressionsanalyse mit Linearisierung oder Ähnliche  nur schwer durchzuführen ist, gibt es schlichtweg mehrere Möglichkeiten die Distanz zwischen Filiale und Kunde zu berechnen.
Die Distanz kann euklidisch dargestellt werden oder über die Netzwerkdistanz oder - zeit (Fahrzeitzone in Meter oder Minuten) wobei alle drei zu verschiedenen Ergebnissen des Modell führen.\\
Weiterführend benutzt das Huff-Modell nur genau zwei Variablen(A und D), die über die Kundenentscheidung des Besuchs entscheiden sollen.
Aufgrund dieser Einschränkung und den bereits aufgeführten Problemen in Bezug auf die Bestimmung der Parameter $\alpha$ und $\beta$ war das Huff-Modell im Laufe der Jahre bereits mehrfach Kern wissenschaftlicher Arbeiten, die sich mit der Validierung und Verbesserung des Modells beschäftigten.\\
Daraus entstand 1974 das (erste \footnote{Es folgten weitere Erweiterungen auf Basis von MCI}) Multiplicative Competitive Interaction Modell (kurz MCI-Modell) entwickelt von Masao Nakanishi und Lee Cooper, welches genau diese Probleme behandeln sollte und eine Bestimmung der Parameter über mathematische Verfahren wie zum Beispiel die (gewöhnliche) Methode der kleinsten Quadrate ermöglichte \footcite{nakanishi_parameter_1974}. 
Weitere wissenschaftliche Modelle sind das Multiple Store Location Model von Dale Achabal \footcite{achabal_slop_1984} oder ein verallgemeinertes Huff-Modell von Daniel Griffith \footcite{griffith_generalized_1982}.
Die Attraktivität einer Filiale hängt von deutlich mehr Faktoren als nur der Größe und Anzahl der Parkplätze ab.
Markenstärke, Sortiment und Preise beeinflussen ebenfalls in die Entscheidung des Kunden, um nur ein paar Weitere zu nennen.
Um diese Faktoren in die Berechnung mit einfließen zu lassen und eine lineare Formel für die Berechnung zu erhalten, verwendet das MCI-Modell folgende Erweiterung der Huff-Modell-Formel:

\begin{equation}
	P_{i j}=\left(\prod_{h=1}^{H} A_{h j}^{\gamma_{h}}\right) D_{i j}^{\lambda} / \sum_{j=1}^{n}\left(\prod_{h=1}^{H} A_{h j}^{\gamma_{h}}\right) D_{i j}^{\lambda}
\end{equation}

Dabei gilt:

\( P_{i j} \) = die Wahrscheinlichkeit, mit der Kunde j in Filiale i einkauft;\\
\( A_{h j} \) = ein Maß der h Eigenschaften (h = 1,2,.H), die die Attraktivität der Filiale j angeben;\\
\( \gamma \) = ein Parameter der Sensibilität von \(P_{i j}\) in Verbindung mit einer Attraktionsvariable h;\\
\( D_{i j}^{\lambda} \) = ein Maß der Erreichbarkeit der Filiale j für einen Kunden i;\\
\( {\lambda} \) = ein Parameter der Sensibilität von \(P_{i j}\) in Verbindung der Erreichbarkeit;\\
n = die Anzahl der Filialen.

Huff selber veröffentlichte diese Formel 2008 in einem Artikel, in dem er die Problematiken des ursprünglichen Modells erläutert und auf die Verbesserungen eingeht und ein Kalibrierung des Modells vorstellt \footcite{huff_calibrating_2008}.

Ebenso finden sich auch direkt im Prototypen einige Schwächen wieder.\\
So lassen sich Filialen finden, bei denen die Berechnung zu einem ungenauen Ergebnis, wie in Abbildung \ref{img:innacc_store} zu sehen, führt.
In der Abbildung ist klar zu erkennen, dass das Gebiet in welchem sich die Filiale befindet selbst nicht die höchste Besuchswahrscheinlichkeit hat.
Diese Ungenauigkeit lässt sich auf die Berechnung des Mittelpunkts des Gebiets zurückführen, so ist die Filiale näher am Mittelpunkt des Nachbargebiets als am Mittelpunkt des eigenen Gebiets gelegen, was zu einer höheren Besuchswahrscheinlichkeit des Nachbargebiets führt.\\
Dieses Problem lässt sich relativ einfach mit der Verwendung von kleineren Gebieten eindämmen.
Jedoch ist die ideale Lösung mit der Verwendung von Punkten pro Haushalten nicht wirklich umsetzbar, da die Datenmenge um ein Vielfaches größer wäre und die Berechnung zu viel Ressourcen benötigen oder zu lange dauern würde.

\begin{figure}[H]
	\makebox[\textwidth][c]{
		\includegraphics[scale=0.4]{resources/images/ungenaue_filiale.png}}
	\caption{Bildschirmausschnitt des Prototypen mit ungenauer Berechnung}
	\label{img:innacc_store}
\end{figure}


\section{Fazit/ Ausblick}
Das Huff-Modell als Grundlage der algorithmischen Berechnung in dem Prozess einer Filialplanung erweist sich als einfach zu verstehendes, schnelles und nützliches Werkzeug, um mit relativ wenig Marktdaten bereits eindeutige und aussagekräftige Informationen über mögliche neue Standorte einer Filiale zu erhalten.\\
Im Prototypen wurden die verfügbaren Marktdaten um lediglich ein paar wenige Attribute ergänzt, um eine realistische Abbildung eines Filialnetzes innerhalb Berlins zu generieren. 
Um die Berechnung komplexer und somit hoffentlich genauer erfolgen zu lassen, ist eine Erweiterung des Huff-Modells wie in \ref{sec:bewertung} beschrieben denkbar.
Die Parameter der Berechnung sollten sich dann einfach fast beliebig Erweitern lassen, so können in die Berechnung der Attraktivität deutlich mehr Attribute einer Filiale oder des Umfelds einfließen.
Denkbar wären zum Beispiel die Ergänzung von Informationen über umliegende Sehenswürdigkeiten (engl. Point of interest), die Eingliederung der Filiale in ein Einkaufszentrum, Markenstärke einer Filialkette oder tatsächliches Sortiment, welche alle Einfluss auf die Attraktivität einer Filiale nehmen.\\
Ebenso kann die Berechnung der Distanz auf komplexeren Grundlagen wie zum Beispiel der Fahrzeit mit dem Auto, den öffentlichen Verkehrsmitteln oder dem Fahrrad basieren.\\
Auf die Notwendigkeit der Bestimmung der Parameter $\alpha$ und $\beta$ wurde bereist im Kapitel \ref{ch:implementierung} Implementierung eingegangen.
Kundenbefragungen, amtliche Statistiken oder Meinungsbilder können hier als Grundlage dienen, um die Parameter zu bestimmen.\\
Für die Anwendung in echten Szenarien der freien Wirtschaft sollte also unbedingt eine verfeinerte Kalibrierung der Berechnung anhand eigener Daten erfolgen, damit der Prototyp zum vollen Potenzial verwendet wird.\\
Außerdem wäre eine Erweiterung hinsichtlich direkter Gegenüberstellung der potenziellen Standorte in Form einer Tabelle oder Ähnlichem denkbar.
Im Prototyp lassen sich zwar beliebig viele neue Filialen setzen, jedoch gibt es keine Möglichkeit die gesetzten Filialen und deren Einfluss auf Wahrscheinlichkeiten und Marktanteil zu vergleichen.\\
Die Darstellung der Gravitationsebenen erfolgt im Prototypen über die Einfärbung der Gebiete.
Eine detaillierte Darstellung direkt über Isolinien der Wahrscheinlichkeit, wäre unter Verwendung einer Geometriefunktion und Stylefunktion möglich. 
So müssten aus den Isolinien neue Polygone erstellt werden und diese anhand der Ebene entsprechend eingefärbt werden.

Als Komponente eines größeren, komplexeren GIS, würde der Prototyp einzigartige, ergänzende Features bieten, die die Prozesse innerhalb der Standortanalyse und Filialplanung deutlich vereinfachen.
Obwohl das Huff-Modell im Laufe der Jahre oft erweitert wurde, führt es in seiner Grundfunktion bereits zu aussagekräftigen Ergebnissen.
Sicherlich können diese Ergebnisse über erwähnte Erweiterungen verbessert und verfeinert werden, jedoch steigt damit auch die Komplexität des Modells und schränkt im Zweifel die allgemeine Anwendung auf spezielle Anwendungsfälle ein.\\
Somit bleibt festzuhalten, durch niedrige Komplexität, vielfältige   Anwendungsbereiche und eine schnelle, simple Umsetzung, bietet sich das Huff-Modell gut für die Anwendung im Prozess der Standortanalyse und Filialplanung an. \clearpage
%\input{chapter/Beispiele} \clearpage

%Anhang
\pagenumbering{Alph}

%Abbildungsverzeichnis
\listoffigures \clearpage
%Tabellenverzeichnis
%\listoftables \clearpage
%Quelltextverzeichnis
\lstlistoflistings \clearpage
%Stichwortverzeichnis
\printindex \clearpage
%Glossar
\printglossary[title={Glossar}] \clearpage
%Abkürzungsverzeichnis
\printglossary[style=dottedlocations,type=\acronymtype,title={Abkürzungsverzeichnis}] \clearpage

%Literaturverzeichnisse (getrennt nach Stichwort)
\printbibliography[title={Literaturverzeichnis}]\clearpage
\printbibliography[heading=bibintoc, keyword={online}, title={Onlinequellen}]\clearpage
\printbibliography[heading=bibintoc, keyword={image}, title={Bildquellen}]\clearpage

% Anhang
\appendix

\chapter{}
\addcontentsline{toc}{chapter}{Anhang A}

\section{Diagramm}

\begin{figure}[ht]
	\centering
	\includegraphics[angle=270, scale=0.35]{resources/images/architecture.png}
		\caption{Architekturbild}
\end{figure}

\begin{figure}[ht]
	\centering
	\includegraphics[angle=270, scale=0.45]{resources/images/Komponentendiagramm.png}
		\caption{Komponentendiagramm}
\end{figure}

\end{document}