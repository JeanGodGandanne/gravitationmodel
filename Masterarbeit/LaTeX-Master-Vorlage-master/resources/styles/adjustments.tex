%Anpassung der Überschriften
\addtokomafont{disposition}{\rmfamily}

%Zusätzliche Farben
\definecolor{darkgreen}{RGB}{0,100,0}

%Umbenennungen
\renewcommand{\lstlistlistingname}{Quelltextverzeichnis}

%Pluszeichen in der Referenc beim zitieren ausblenden
\renewcommand*{\labelalphaothers}{}

%Anpassugen zur Quelltextdarstellung, kann bei Bedarf überschrieben werden (z.B. wenn unterschiedliche Sprachen zum Einsatz kommen)
\renewcommand{\lstlistingname}{Codeauszug}
\lstset{
	numbers=left,
	columns=fullflexible,
	aboveskip=5pt,
	belowskip=10pt,
	basicstyle=\small\ttfamily,
	backgroundcolor=\color{black!5},
	commentstyle=\color{darkgreen},
	keywordstyle=\color{blue},
	stringstyle=\color{gray},
	showspaces=false,
	showstringspaces=false,
	showtabs=false,
	xleftmargin=16pt,
	xrightmargin=0pt,
	framesep=5pt,
	framerule=3pt,
	frame=leftline,
	rulecolor=\color{green},
	tabsize=2,
	breaklines=true,
	breakatwhitespace=true,
	prebreak={\mbox{$\hookleftarrow$}}
}

\lstdefinelanguage{CSS}{
	keywords={color,background-image:,margin,padding,font,weight,display,position,top,left,right,bottom,list,style,border,size,white,space,min,width, transition:, transform:, transition-property, transition-duration, transition-timing-function},	
	sensitive=true,
	morecomment=[l]{//},
	morecomment=[s]{/*}{*/},
	morestring=[b]',
	morestring=[b]",
	alsoletter={:},
	alsodigit={-}
}

% JavaScript
\lstdefinelanguage{JavaScript}{
	morekeywords={typeof, new, true, false, catch, function, return, null, catch, switch, var, if, in, while, do, else, case, break},
	morecomment=[s]{/*}{*/},
	morecomment=[l]//,
	morestring=[b]",
	morestring=[b]'
}

\lstdefinelanguage{HTML5}{
	language=html,
	sensitive=true,	
	alsoletter={<>=-},	
	morecomment=[s]{<!-}{-->},
	tag=[s],
	otherkeywords={
		% General
		>,
		% Standard tags
		<!DOCTYPE,
		</html, <html, <head, <title, </title, <style, </style, <link, </head, <meta, />,
		% body
		</body, <body,
		% Divs
		</div, <div, </div>, 
		% Paragraphs
		</p, <p, </p>,
		% scripts
		</script, <script,
		% More tags...
		<canvas, /canvas>, <svg, <rect, <animateTransform, </rect>, </svg>, <video, <source, <iframe, </iframe>, </video>, <image, </image>, <header, </header, <article, </article
	},
	ndkeywords={
		% General
		=,
		% HTML attributes
		charset=, src=, id=, width=, height=, style=, type=, rel=, href=,
		% SVG attributes
		fill=, attributeName=, begin=, dur=, from=, to=, poster=, controls=, x=, y=, repeatCount=, xlink:href=,
		% properties
		margin:, padding:, background-image:, border:, top:, left:, position:, width:, height:, margin-top:, margin-bottom:, font-size:, line-height:,
		% CSS3 properties
		transform:, -moz-transform:, -webkit-transform:,
		animation:, -webkit-animation:,
		transition:,  transition-duration:, transition-property:, transition-timing-function:,
	}
}

\lstdefinestyle{htmlcssjs} {%
	% General design
	%  backgroundcolor=\color{editorGray},
	basicstyle={\footnotesize\ttfamily},   
	frame=b,
	% line-numbers
	xleftmargin={0.75cm},
	numbers=left,
	stepnumber=1,
	firstnumber=1,
	numberfirstline=true,	
	% Code design
	identifierstyle=\color{black},
	keywordstyle=\color{blue}\bfseries,
	ndkeywordstyle=\color{editorGreen}\bfseries,
	stringstyle=\color{editorOcher}\ttfamily,
	commentstyle=\color{brown}\ttfamily,
	% Code
	language=HTML5,
	alsolanguage=JavaScript,
	alsodigit={.:;},	
	tabsize=2,
	showtabs=false,
	showspaces=false,
	showstringspaces=false,
	extendedchars=true,
	breaklines=true,
	% German umlauts
	literate=%
	{Ö}{{\"O}}1
	{Ä}{{\"A}}1
	{Ü}{{\"U}}1
	{ß}{{\ss}}1
	{ü}{{\"u}}1
	{ä}{{\"a}}1
	{ö}{{\"o}}1
}

\crefformat{footnote}{#2\footnotemark[#1]#3}

%Anpassungen für das Abkürzungsverzeichnis
\newglossarystyle{dottedlocations}{%
	\renewcommand*{\glossaryentryfield}[5]{%
		\item[\glsentryitem{##1}\glstarget{##1}{##2}] \emph{##3}%
		\unskip\leaders\hbox to 2.9mm{\hss.}\hfill##5}%
	\renewcommand*{\glsgroupskip}{}%
}