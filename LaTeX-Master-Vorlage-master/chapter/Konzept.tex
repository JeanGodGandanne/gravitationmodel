\chapter{Konzept}
Die Algorithmen und Prozesse einer Filialplanung spiegeln sehr gut einen allgemeinen Anwendungsfall moderner GIS. 
Neben einer einfachen Hintergrundkarte sind die Mitarbeiter auf Zeichenwerkzeuge, Geo-Daten-Anzeige, thematische Karten und Kennzahlen Berechnungen angewiesen, um akkurate und fundierte Prognosen und Planungen treffen zu können.
Viele theoretische Konzepte und Prozesse passieren hierbei im Hintergrund und sind nicht direkt ersichtlich.
In den folgenden Kapiteln werden die wichtigsten Beschrieben mit dem Fokus auf den Algorithmus der hinter dem Gravitations-Model von Huff steht.
\section{Algorithmen und Prozesse}
Um eine realistische Abbildung der Welt auf eine Karte zu bringen benötigt es eine Umrechnung eines Ovales, quasi der Erdkugel, auf eine Rechteck, im Fall eines Web-Gis des Bildschirms. 
\subsection{Gravitations-Modelle}
Das \textit{Huff-Modell} (engl. Huff Gravitation Model) ist ein mathematisches Modell zur Abgrenzung und Segmentierung von Marktgebieten \cite{Roy2004}.
Das Modell bestimmt die Wahrscheinlichkeit, mit der Kunden einen Standort (Filiale, Einkaufszentrum) in Abhängigkeit von Distanz und Attraktivität aufsuchen. 
Die Wahrscheinlichkeiten werden nun auf der Karte als Punkte dargestellt, die rund um den Standort platziert werden. 
Die Punkte der selben Wahrscheinlichkeiten werden zu Isowahrscheinlichkeitslinien verbunden. 
So entstehen zu jedem Standort verschiedene Gravitationsebenen, die farblich gekennzeichnet werden. 
Die Beeinflussung der verschiedenen Gravitationsebenen führt zu mehreren Farbverläufen, die ein komplexes Bild der Standort Landschaft bilden.
In Seiner einfachsten Form berücksichtigt das Modell nur die Distanz und eine simple Kennzahl der Attraktivität (zum Beispiel in Form eines Rankings) für die Wahrscheinlichkeitsberechnung. 
Für die komplexere Berechnung mit mehr Faktoren wird das MCI-Modell von Nakanishi und Cooper verwendet, welches die Weiterentwicklung der einfachen Huff-Modells darstellt \cite{mciModell}.

Das \textit{MCI-Modell} ...
\section{Architektur}
